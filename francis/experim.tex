%!TEX root = francis_thesis.tex
%%%%%%%%%%%%%%%%%%%%%%%%%%%%%%%%%%%%%%%%%%%%%%%%%%%%%%%%%%%%%%%%%%%%%%%
\chapter{Conclusion}\label{ch:PRACTICE}

\begin{summary}
The aim of this project is to automatically detect and segment objects in a drone-based dataset by extending the application of Mask R-CNN the drone-based dataset. With the advance of the applicability of drone to daily-life activities, industries and farms. Counting objects with drones is a new application of drone. This work aim to provide a system for object detection in drone-based dataset that can be practically applied in facilitating work performed by industries like agricultural, oil and gas industries and so on.  As a future work the model will be deployed in the drone developed by the Robotic Team of African University of Science and Technology (AUST) for detection and control of pipeline vandalism in oil and gas industries. 
The first part of the thesis was dedicated to a theoretical background behind
CNNs.It also discussed by the general overview of various computer vision tasks. The second part is dedicated to the introduction of tools used in the work and the implementation of Mask R-CNN modules using the technologies on drone-based datasets. More importantly explanation to most important parts of the code.
Developed modules are available in GitHub repository.  


\end{summary}


%%%%%%%%%%%%%%%%%%%%%%%%%%%%%%%%%%%%%%%%%%%%%%%%%%%%%%%%%%%%%%%%%%%%%%%


