%!TEX root = aust_thesis.tex
%%%%%%%%%%%%%%%%%%%%%%%%%%%%%%%%%%%%%%%%%%%%%%%%%%%%%%%%%%%%%%%%%%%%%%%
\chapter{Introduction}\label{ch:INTRO}
%%%%%%%%%%%%%%%%%%%%%%%%%%%%%%%%%%%%%%%%%%%%%%%%%%%%%%%%%%%%%%%%%%%%%%%

\begin{summary}
	 The summary at the beginning of each chapter explains what is going to be written in that chapter.
\end{summary}

%%%%%%%%%%%%%%%%%%%%%%%%%%%%%%%%%%%%%%%%%%%%%%%%%%%%%%%%%%%%%%%%%%%%%%%
\section{First section}\label{sec:INTRO:ml}

Here the description of the concepts relating to your work.%
\footnote{ %
The test of the files are examples in Hungarian related to the EDITIING of the thesis. In what follows there is a CITATION: \citeN{Mitchell97}:\newline
    ``The field of machine learning is concerned with the question of how to construct computer programs    that automatically improve with experience.''
}




\begin{figure}[t]
  \centering
  \pgfimage[width=0.2\linewidth]{images/bayes}
  \caption[Example figure]%
  {Example of including a figure. Please always state its origin if NOT YOUR OWN:\\
  {\white .}\hfill\url{http://en.wikipedia.org/wiki/Thomas_Bayes}}
  \label{fig:ALAP:sm1}
\end{figure}

Including figures:\\
 \verb+\pgfimage[width=0.4\linewidth]{images/bayes}+\\
where parameters:\\
\verb+width=0.4\linewidth+ \\

Putting images side by side using \verb+tabular+


If one puts graphics, illustrations, it is better to use the VECTOR format -- we mean PDF or EPS, eventually SVG -- as one sees in Fig.~\ref{fig:ALAP:sm3}

\begin{figure}[t]
  \centering
  \begin{tabular}{ccc}
		  \pgfimage[height=4cm]{images/bayes}
		  &
		  \pgfimage[height=4cm]{images/vapnik}
	\end{tabular}
  \caption[Side by side example]%
  {Two eminent scientists.\\
  {\white .}\url{http://en.wikipedia.org/wiki/Vladimir_Vapnik}}
  \label{fig:ALAP:sm2}
\end{figure}

\begin{figure}[t]
  \centering
  \pgfimage[width=0.7\linewidth]{images/parzen}
  \caption[Vector graphics example]%
  {Caption for the vector graphics example.}
  \label{fig:ALAP:sm3}
\end{figure}

It is also recommended the usage of:
\begin{itemize}
	\item elements like: \verb+\chapter+, \verb+\section+, \verb+\subsection+, \verb+\subsubsection+
	\item use of different types of lists;
\end{itemize}