%!TEX root = kudzai_thesis.tex
%%%%%%%%%%%%%%%%%%%%%%%%%%%%%%%%%%%%%%%%%%%%%%%%%%%%%%%%%%%%%%%%%%%%%%%
\chapter{Introduction}\label{ch:INTRO}
%%%%%%%%%%%%%%%%%%%%%%%%%%%%%%%%%%%%%%%%%%%%%%%%%%%%%%%%%%%%%%%%%%%%%%%

%%%%%%%%%%%%%%%%%%%%%%%%%%%%%%%%%%%%%%%%%%%%%%%%%%%%%%%%%%%%%%%%%%%%%%%
Sentiment analysis makes use of computational techniques to study peoples' emotions and opinions
on given topics. In recent years this field has attracted a lot of attention from both academia and industry, it comes with a lot of challenging research problems but has a wide range of applications.

Whenever we want to make a decision we must take into consideration the opinions of others, this is what makes opinions important. Both individuals and organizations who want to know the opinions of others benefit from this.

Prior to the web,  no computational study on peoples opinions was being done. Opinionated text did not exist in abundance. To get peoples opinion one would typically need to use techniques such as surveys or questionnaires to get opinions from the public or simply ask from friends and or family members.
When organizations wanted to get opinions about services or products they would typically use these methods.   

But due to the explosive growth of social media websites and mobile applications, opinionated content on the web has increased exponentially. People can now share their opinions about almost anything on blogs, comment sections and social websites\cite{ref47}





